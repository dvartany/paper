\documentclass{article}
\usepackage[hang,bf,small]{caption}
\usepackage{hyperref}
\usepackage[all]{hypcap}

\setlength{\topmargin}{-.5in}
\setlength{\textheight}{9in}
\setlength{\oddsidemargin}{.125in}
\setlength{\textwidth}{6.25in}
\usepackage{graphicx}
\usepackage{caption}
\usepackage{amsmath}
\usepackage{subcaption}
\begin{document}

\section{Introduction}

\subsection{LGRB-SN Connection}

\subsection{Magnetar Model}

\subsection{Collapsar Model}

\section{Magnetar Spindown as a Source of GRB Engine Power}
In this section we will explore the fundamental physics of magnetar spindown as it relates to both the prompt emission and the afterglow phases of the LGRB light curve. In Section \ref{sec:magnetization} introduce the magnetization parameter $\sigma$, relevant for our subsequent discussion of the prompt emission energetics and duration. In Section \ref{sec:dipole} we will then derive the dipole model for magnetars which will be relevant for our discussion of afterglow plateaus. Finally, in Section \ref{sec:fallback} we will follow Piro and Ott 2011 \cite{Piro:2011ed} to account for possible fallback accretion onto our magnetar. This may affect both prompt emission and afterglow energetics and duration by modifying the magnetization parameter and the magnetar period.

\subsection{Magnetization} \label{sec:magnetization}

The magnetization parameter is defined as: 
\begin{equation}
\sigma_o= \frac{\phi^2 \Omega^2}{\dot{M} c^3}
\end{equation}

Naively, we can think of this as a ratio of magnetic field energy density to mass energy density. As will be discussed in Section ~\ref{sec:prompt}, the magnetization parameter increases drastically on a timescale of 20-100s, during which the neutron star becomes optically thin to neutrinos and consequently $\dot{M}$ declines. This timescale constrains prompt emission duration since jets with high magnetization cannot effectively accelerate and dissipate their energy \cite{Metzger:2010pp}. The transparency timescale is tauntingly similar to the duration of typical LGRBs of $\sim 100$ seconds. Thus, we search for physical processes that may decrease the magnetization. Barring decreasing the magnetix flux or the angular frequency, which would also counterproductively decrease the spindown energy, or increasing the speed of light, which stubbornly chooses to remain constant,  we are left with increasing $\dot{M}$. Then, the pressing question is whether magnetization is affected only by mass loss or mass accretion as well. We ask:

\begin{itemize}
\item Can fallback accretion increase $\dot{M}$ and hence lower magnetization to allow for longer prompt emission winds?
\item Does the wind direction matter? Metzger 2010 \cite {Metzger:2010pp} solves for prompt emission energetics assuming neutrino-driven mass loss (an outflow). Can we similarly account for accreting mass from the supernova (an inflow) in calculationg the magnetization? If not:
\item Following Piro and Ott 2011 \cite {Piro:2011ed}, there exist certain conditions under which fallback accretion is flung back out. In this propeller regime, we would have mass loss in that sense that mass is flowing outwards from the magnetar, addressing the previous issue. However, compared to Metzger 2010, where neutrino driven mass-loss happens at the surface of the magnetar, the propeller regime occurs at the Alfven radius of $\sim 14$ km (see Section \ref{sec:fallback}). Is this difference significant, given that the internal shock radius responsible for prompt emission is $\sim 10^{13}$ cm and thus much further than both the Alfven radius and the magnetar surface?
\end{itemize}

\subsection{Magnetar Dipole Radiation} \label{sec:dipole}

We model magnetars similarly to pulsars and assume a magnetic dipole toy model. The dipolar magnetar field is:

\begin{equation} B(\vec{r})=\frac{3\vec{n}(\vec{m}.\vec{n})-\vec{m}}{r^3}, \end{equation}
where $\vec{m}$ is the magnetic moment and $\vec{n}$ is the unit radial vector.

In analogy with Larmor's formula for electric dipole radiation, a time-dependent magnetic dipole radiates

\begin{equation}
\frac{dW}{dt} = -\frac{2}{3c^3} *{\ddot{|\vec{m}}_\perp|}^2,
\end{equation}

where $\vec{m_\perp}$ is the component of $\vec{m}$ perpendicular to rotation axis.

Defining the angle between the rotation axis and the magnetic dipole moment as $\alpha$,

\begin{equation}\vec{m}_\perp = m_o  \sin({\alpha}) e^{-Iwt},\end{equation}
so $\ddot{\vec{|m|}}_\perp^2 = {m_o}^2 \sin{\alpha}^2 w^4$
since $m_o = B R^3/2$ for a uniformly magnetized sphere.

It follows that \begin{equation} \label{eq:lumin}\frac{dW}{dt} = -\frac{{B_p}^2 R^6}{6c^3} w^4 \sin^2{\alpha},\end{equation}

If we assume the magnetic dipole is oriented perpendicularly to the rotation axis so $\alpha=\pi/2$, the luminosity is powered by spin-down. We define the magnetar spin period $P= 2\pi/\omega$ and arrive at equation 2 from Lyons 2010:

\begin{equation} L  = 9.62065*10^{48}\, B^2_{p,15} P^{-4}_{-3} R^6_ 6\,\mathrm{erg \,s^{-1}},\end{equation}
where $B_{p,15} = B_p/10^{15}$, etc.

Next we assume dipole radiation taps the rotational energy of the magnetar, so $\frac{dE_{rot}}{dt} = \frac{dW}{dt}$ where $E_{rot} = 1/2 I w^2$ so $\ddot{E}_{rot} = I \omega \ddot{\omega}$. Define a characteristic dipole spindown time $\tau_{dipole}$ as

$\tau_{\mathrm{dipole}} = - \omega/\ddot{\omega}$
It follows that

\begin{equation}\tau_{\mathrm{dipole}} = \frac{3c^3 I}{{B_p}^2 R^6 \omega^6},
\end{equation}

Then, \begin{equation}\tau_{\mathrm{dipole}} = 2051.75\,\, I_{45} B^{-2}_{15,p} P^2_{-3} R_6^{-6}\,\mathrm{s},\end{equation}  which is equation 3 in Lyons 2009 \cite{Lyons:2009ka}.

Lyons assumes $P=P_o$, using initial period instead and neglecting spindown.

\subsection{Fallback Accretion and the Propeller Mechanism} \label{sec:fallback}

We are interested in solving for the magnetar period evolution in the presence of fallback accretion. Thus, we assume a weaker supernova leaving the magnetar in a nonvacuum environment.  We will follow Piro and Ott 2011 \cite{Piro:2011ed} in the following. \smallskip

Accretion falls under the magnetar's field influence at the Alfven radius,

\begin{equation}
r_m=\mu^{4/7}(G M)^{-1/7} \dot{M}^{-2/7}
\end{equation}

Material will corotate with the magnetar up to the cororation radius,
\begin{equation}
r_c=\frac{GM}{\Omega^2}^{1/3}
\end{equation}

If $r_m>r_c$, infalling material will come under the dipole field's influence, spin at a super-Keplerian rate, and be flung out. If $r_m<r_c$, material will come under the dipole field's influence at a sub-Keplerian rate and be funneled onto the magnetar. This delineates the accretion regime from the propeller regime.

Then, we can solve for the magnetar period dynamics by conserving angular momentum:

\begin{equation}
I \frac{d\Omega}{dt} = N_{dip} + N_{acc}
\end{equation}

where
\begin{equation}
N_{dip}= -\frac{\mu^{2}\Omega^{3}}{6 c^{3}}
\end{equation}

We divide $N_{acc}$ into two cases: when $r_m>R$ and and when $r_m<R$. Only in the first case does infalling material come under the dipole field's influence.

We then have:
\begin{equation}
N_{acc}= n(\omega) (GMr_m)^{1/2} \dot{M}\, \,  \rm{for \,\, r_m>R}
\end{equation}

where $n(\omega)=(\frac{\Omega}{GM/r_m^3})^{1/2}=(r_m/r_c)^{3/2}$. $n(\omega)$ is chosen such that $N_{acc}$ is positive, spinning up the magnetar, in the accretion regime $r_m<r_c$ and negative, spinning down the magnetar, in the propeller regime, $r_m>r_c$. For an Alfven radius internal to the magnetar, $r_m<R$, we have:
\begin{equation}
N_{acc}=(1-\frac{\Omega}{\Omega_k})(GMR)^{1/2} \dot{M}\, \,  \rm{for \,\, r_m<R}
\end{equation}.
Infalling material does not come under the influence of the magnetic field before accretion. The prefactor ensures continuity of $N_{acc}$ at $r_m=R$.

The mass accretion rate can be decomposed into early and late times:

\begin{equation} \label{eq:early}
\dot{M}_{early} = \eta 10^{-3}t^{1/2} M_{\odot} s^-1
\end{equation}
\begin{equation} \label{eq:late}
\dot{M}_{late}=50 t^{-5/3}M_{\odot} s^{-1}
\end{equation}

Following the convention in Piro and Ott 2011, we combine these expressions
\begin{equation}
\dot{M}=(\dot{M}_{early}^{-1}+\dot{M}_{late}^{-1})^{-1}
\end{equation}
which preserves the time domains of our decomposed accretion rate equations \ref{eq:early} and \ref{eq:late}.
\section{Prompt Emission and Magnetic Dissipation} \label{sec:prompt}

\subsection{Magnetization and Longevity}

\subsection{Thermalization: Dissipation vs Shocks}

\section{Afterglow and Plateau}

\subsection{Lyons: Spindown Powered Plateau}

\subsubsection{Assumptions}

\subsubsection{K-Correction}

\subsection{Centrifugal Support and Black Hole Formation}

\section{Conclusions}

\bibliographystyle{unsrt}
\bibliography{thesisoutline}

\end{document}
